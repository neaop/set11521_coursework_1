\documentclass[twocolumn]{article}
\usepackage{apacite}
\begin{document}
\title{SET11521 - Coursework Part 1}
\author{40056761}
\date{16/03/2018}
\maketitle

\section{Introduction}
Sentiment analysis (SA) is a Natural Language Processing (NLP)
technique used to identify the opinion of a
body of text, i.e. whether the meaning being conveyed is positive or negative.

SA has been proven to perform effectively upon traditional
texts, but current interest is focused on the analysis of informal texts in
the form of social media.

Companies, academics, and governments are all embracing social media as a forum
to collect feedback and track cultural shifts. 
This makes SA of social media a particularly 'trendy' area of
research, as there are a range of stakeholders and approaches in one domain.

This paper records a review and comparison of three different sentiment
analysis systems used to categorise opinions from tweets.
The systems include: A supervised solution \cite{Kaur2015}, a lexicon
approach \cite{Muhammad2016}, and a combined hybrid system
\cite{Asghar2017}.

\section{Background}
NLP is a field of computing that focuses on systems understanding human
language.  

SA is a subdomain of NLP that focuses on extracting the sentiment from text.
SA is typically sorts text into categories - positive or negative.

There are two main SA techniques;
Knowledge-based and Statistical based.

Knowledge-based approaches require the use of pre- existing data.
The most basic form of knowledge-based SA consists of
comparing the words in a body of text (corpus) to those in an annotated
dictionary (lexicon).
A lexicon contains a set of words and the sentiment each word conveys -
typically a value between -1 (very negative) and 1 (very positive).
A naive approach to SA is to simply add together the lexicon
scores for each word in a corpus.
While such a system may be simple to develop, it is unlikely to produce
accurate results without incorporating other techniques.

Some techniques used to improve the effectiveness of knowledge-based systems
include;

Tokenisation - splitting a text into multiple tokens/n-grams. 
Part-of-speech tagging (PoS) - categorising words as nouns, verbs, etc.

Lemmatisation - reducing a word to its base meaning.
Stop word removal - removing words that carry no meaning - 'that', 'a', 'the',
etc.

Statistical approaches to SA often rely on machine learning
(ML) techniques in their attempt to classify a corpus.
Statistical approaches utilise one of two styles of technique;
supervised learning, or unsupervised learning.

Supervised learning consists of techniques that train a system to categorise
input by first training it with a pre-labelled data set.
A method to conduct supervised learning would to be to hand categorise a
number of tweets as either positive, negative or neutral and use these to
train a system. 
Once trained the system could attempt to categorise new corpora
by attempting to match features that were present in the training data.

Supervised learning has been shown to be effective method of classifying a
range of different data, but also has a number of drawbacks. 
A substantial amount of training data must be gathered and categorised
which can be a time consuming and monotonous.

A second potential disadvantage of a supervised system is that it is possible to
over-train.
Over-training occurs when the data used to train the system is too dissimilar
to the data used to test it. 
This can occur due to training set not being diverse in quality or style, or
when training data been overly cleaned and no longer representative of
the testing data.

Unsupervised learning is a ML technique that operates on unlabelled data.
Unsupervised learning is considered a black-box, as there is essentially no
way of determining which data features the system could use to classify
inputs.
Unsupervised techniques such as clustering and neural networks can be
attractive approaches to solving classification problems, as large data sets
do not have to be hand labelled.
Such systems may also be able to exploit patterns in data that are unintuitive
or unrecognisable to humans.

\section{Discussion}
While the three systems being reviewed in this paper each had the same goal -
the sentiment classification of tweets - the approaches and methodologies of
each system varied greatly.  
This section aims to highlight and compare these differences.

\subsection{Data-sets}
Each system was tested with a bespoke dataset, rather than a pre-existing
published  corpus of tweets.  
This was required as two of the system were designed to be used within
specific domains or genres, and a general purpose dataset would not be
suitable.

The supervised system compiled its dataset from tweets posted between September
and October 2014 which contained the hashtags: \#kashmirfloods and \#jammufloods
via use of the Twitter API.
8490 tweets were collected, of which 6165 made up the final dataset.
Once the tweets had been gathered the dataset had to go through preprocessing
to make it suitable for the system.
The tweets were converted to lower-case, punctuation was removed as were extra
characters - e.g. the word 'saaad' became 'sad'.  
Tokenisation and PoS tagging were applied to the tweets and common stop words
were removed. 
As the floods mentioned in the tweets took place in India, many of the tweets
were written in Hindi.
Rather than ignoring non-English tweets, they were translate to English via
Google Translate.  
Once the dataset was cleaned, it was categorised and annotated by hand,
resulting in 1537 positive, 917 negative, and 3711 neutral tweets.
The dataset was then split into training and testing sets, containing 5549 and
616 tweets, respectively.

The lexicon system was designed to classify social media post of multiple
formats, including Twitter, MySpace and Digg.
The Twitter data set for this system was collected via use of sentiment140.com
- a service that aggregates tweets containing specific keywords and apply its
own sentiment classification annotations.
In total 20000 tweets were collected for the dataset, 10000 positive and 10000
negative.
Preprocessing was then applied to the dataset; the Tweet NLP library was used
to tokenize and apply PoS tags to each of the tweets.
Lemmasation was then applied to the tweets to replace words with their
dictionary root.

The hybrid system utilised the Tweepy Python library to gather tweets.
Three datasets were collected using key words from three different domains; 
auto mobile, laptops, and health.
The IBM  AlchemyAPI was used to automatically categorize tweets into
positive, negative and neutral categories - resulting in 10364 tweets across
three distinct subject genres.
The Hybrid system's datasets went through a thorough preprocessing;
the Python NLTK library was used to tokenise each tweet and Lemmatation and
PoS tags were also added.  
Finally, negation words ('but', 'not', etc.) were identified so as to reduce
or invert the sentiment of neighbouring words.
10364 tweets were colleceted, split into training and test sets at a ratio of
9:1.

As each approach required a unique dataset, it makes comparison between
papers difficult.
It appears that a supervised approach was overzealous during its data
preprocessing, removing many sentiment carrying features that are unique to
social media e.g. emoticons, capitalisation, and repeating letters - resulting
in a dataset very similar to a traditional text.
Another issue with the Supervised dataset was the choice to translate
non-English tweets via Google Translate, potentially loosing all form
sentiment during the conversion process.
Conversely, both the lexicon and hybrid system's datasets retained much of the
social-media specific features of the Tweets but did not manually label and
categorise there data - opting to utilise pre-existing SA systems.
While the choice to automate data labelling would have saved significant time,
it is likely that a percentage of the data would be mislabelled,
leading to inaccurate results.

\subsection{Approach}
As each system utilised different domain methodologies, each  of the
papers approached the task of SA differently.

The supervised system operated via a Naive Bayes classifier algorithm.
Naive Bayes is a simple probabilistic classifier that is based on Bayes
theorem.
Essentially, the algorithm will attempt to classify a word based upon the
words rate of occurrence in tweets of a particular category - e.g. if the word
'sad' appeared in 50% of negative tweets and in only 5% of positive tweets, the
algorithm would classify 'sad' as a negative word.
A Naive Bayes algorithm requires experience in order to accurately classify
tokens - hence the reason the dataset was split into two - the larger set used
to train and the smaller to test.
The algorithm was trained and tested using unigrams (a single token at a
time).

The lexicon approach did not rely on a single general purpose lexicon. A
second lexicon was generated for domain specific words, i.e. words that the
general lexicon had no entry for or may have a differing polarity based on the
context they are used within.
SentiWordNet (SWN) was used as the general purpose lexicon in this system.
SWN is a free to use lexicon that has been used in numerous NLP and SA
systems.
A domain-focused (DF) lexicon was automatically generated by scanning the
dataset for unique terms, and assigning a polarity to said word based on the
annotated sentiment of the tweet and any sentiment carrying emoticons.
A hybrid lexicon was then generated by combining the SWN and DF lexicons.
In addition to the hybrid lexicon, this system  used a number of rules in
order to better categorise sentiment based on the context of a sentence.
Lexical polarity shifters (words like 'very', or 'slightly') and negators
('but, 'never') alter the sentiment of words around them - increasing,
decreasing, reversing or simply removing polarity depending on the word.
Non-lexical sentiment rules were also incorporated into the system;
emoticons, capitalisation, and letter repetition were all analysed.
Interestingly, the logic of the system would cause any tweet that contained a
single emoticon to be assigned the sentiment of said emoticon - an entirely
negative tweet that ended with a grinning face ('{\tt:D}') would be considered
highly positive.

The hybrid system combined elements from both knowledge-based and statistic
domains of SA.
The knowledge-based elements consisted of a number of lexicons, SWN, slang and
emoticon lexicons were all used to perform general lexical analysis of the
corpus.
In addition, a domain specific lexicon was generated for each dataset domain
via the use of a Naive Bayes algorithm in a similar style to the supervised
approach.
Words that occurred repeatedly but had no entry in the SWN lexicon, or were
found to be in tweets of different sentiment classification were added to a
DF lexicon, with a polarity score based upon average polarity of the tweets
that they were from.
While this proved to be an effective way to collect domain specific words,
it did result in certain words that should carry no sentiment being assigned
an incorrect polarity - the name 'Suzuki' appeared in many positive tweets, but
a brand name is not a subjective term.

Each systems approach to the SA task varied not only in style but also in
complexity.
The supervised approach utilised one of the most simple classification
functions available, while the lexicon approach had a distinct (if not
confusing) focus on sentence structure and analysing context.
The hybrid system's creative approach to lexicon generation via machine
learning allowed the system to combine two relatively simple concepts without
over complicating either, resulting in a system that was nuanced but not
convoluted.

\subsection{Results}
While the three systems implemented different NLP techniques, they each
analysed their results via a confusion matrix.
A confusion matrix tracks the number of correct and incorrect classifications
and allows the evaluation of a system's accuracy - the F-score is a normalised
value between 0 and 1, that provides a balanced result taking into account
both the precision and recall of a system.
The average F-scores for each system are presented in table \ref{my-label}:

\begin{table}[h]
\centering
\caption{F-scores of each system}
\label{my-label}
\begin{tabular}{l|r}
\multicolumn{1}{c|}{System} & \multicolumn{1}{c}{F-score} \\ \hline
Supervised                  & 0.55                        \\
Lexicon                     & 0.71                        \\
Hybrid                      & 0.78                       
\end{tabular}
\end{table}

It is worth questioning the validity of the lexicon and hybrid system's
results, as both used automatically labelled datasets, which may have skewed
performance.

It is also noteworthy that the Supervised system, which removed many
social-media specific features from its corpus, produced the lowest F-score.
Considering the dataset of the system was extremely similar to a traditional
corpus this result is unintuitive.

Both the lexicon and hybrid systems utilised general purpose and DF lexicons to
classify tweets.
The hybrid system was able to achieve a higher F-score, and this may be due to
the more advanced approach used to create its DF lexicons.

The lexicon system had an unusual obsession with emoticon usage, generating
its DF lexicon and classifying entire tweets based on the presence of
emoticons.
This naive reliance would likely result in incorrect classifications, due to
satirical or improper use of emoticons in the corpus.

\section{Conclusion}

There are numeral ways to approach the task of SA, and the papers reviewed in
this report were chosen to reflect these choices.

It is clear from the results that the systems that utilised more advanced
features, and took into account the unique nature of social media's language
performed to a higher degree.
While it may just be an artefact of the automatically labelled datasets, the
lexicon and hybrid systems both had a focus on non-lexical component's,
leading to a higher F-score compared to the supervised system.

Ideally, an optimal SA investigation would combine the hand-labelled dataset
of the supervised system (without over-cleaning) and the dual lexicon and
machine learning  approach of the hybrid system to thoroughly examine if the
hybrid system truly outperformed the other approaches presented in this paper.

\bibliography{bibl.bib}
\bibliographystyle{apacite}

\end{document}

